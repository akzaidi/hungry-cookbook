\PassOptionsToPackage{unicode=true}{hyperref} % options for packages loaded elsewhere
\PassOptionsToPackage{hyphens}{url}
%
\documentclass[]{article}
\usepackage{lmodern}
\usepackage{amssymb,amsmath}
\usepackage{ifxetex,ifluatex}
\usepackage{fixltx2e} % provides \textsubscript
\ifnum 0\ifxetex 1\fi\ifluatex 1\fi=0 % if pdftex
  \usepackage[T1]{fontenc}
  \usepackage[utf8]{inputenc}
  \usepackage{textcomp} % provides euro and other symbols
\else % if luatex or xelatex
  \usepackage{unicode-math}
  \defaultfontfeatures{Ligatures=TeX,Scale=MatchLowercase}
\fi
% use upquote if available, for straight quotes in verbatim environments
\IfFileExists{upquote.sty}{\usepackage{upquote}}{}
% use microtype if available
\IfFileExists{microtype.sty}{%
\usepackage[]{microtype}
\UseMicrotypeSet[protrusion]{basicmath} % disable protrusion for tt fonts
}{}
\IfFileExists{parskip.sty}{%
\usepackage{parskip}
}{% else
\setlength{\parindent}{0pt}
\setlength{\parskip}{6pt plus 2pt minus 1pt}
}
\usepackage{hyperref}
\hypersetup{
            pdftitle={The Hungry Cookbook},
            pdfauthor={Ali Zaidi and Sampi Chan},
            pdfborder={0 0 0},
            breaklinks=true}
\urlstyle{same}  % don't use monospace font for urls
\usepackage[margin=1in]{geometry}
\usepackage{longtable,booktabs}
% Fix footnotes in tables (requires footnote package)
\IfFileExists{footnote.sty}{\usepackage{footnote}\makesavenoteenv{longtable}}{}
\usepackage{graphicx,grffile}
\makeatletter
\def\maxwidth{\ifdim\Gin@nat@width>\linewidth\linewidth\else\Gin@nat@width\fi}
\def\maxheight{\ifdim\Gin@nat@height>\textheight\textheight\else\Gin@nat@height\fi}
\makeatother
% Scale images if necessary, so that they will not overflow the page
% margins by default, and it is still possible to overwrite the defaults
% using explicit options in \includegraphics[width, height, ...]{}
\setkeys{Gin}{width=\maxwidth,height=\maxheight,keepaspectratio}
\setlength{\emergencystretch}{3em}  % prevent overfull lines
\providecommand{\tightlist}{%
  \setlength{\itemsep}{0pt}\setlength{\parskip}{0pt}}
\setcounter{secnumdepth}{5}
% Redefines (sub)paragraphs to behave more like sections
\ifx\paragraph\undefined\else
\let\oldparagraph\paragraph
\renewcommand{\paragraph}[1]{\oldparagraph{#1}\mbox{}}
\fi
\ifx\subparagraph\undefined\else
\let\oldsubparagraph\subparagraph
\renewcommand{\subparagraph}[1]{\oldsubparagraph{#1}\mbox{}}
\fi

% set default figure placement to htbp
\makeatletter
\def\fps@figure{htbp}
\makeatother

\usepackage{booktabs}
\usepackage{makeidx}
\makeindex
\usepackage[]{natbib}
\bibliographystyle{apalike}

\title{The Hungry Cookbook}
\author{Ali Zaidi and Sampi Chan}
\date{2020-07-20}

\begin{document}
\maketitle

% you may need to leave a few empty pages before the dedication page

\begin{center}
For the Hungry. \\
By the Hungry.

\end{center}

{
\setcounter{tocdepth}{2}
\tableofcontents
}
\hypertarget{welcome}{%
\section*{Welcome!}\label{welcome}}
\addcontentsline{toc}{section}{Welcome!}

\hypertarget{part-chicken-recipes}{%
\part{Chicken Recipes}\label{part-chicken-recipes}}

\hypertarget{peri-peri-marinade}{%
\section{Peri Peri Marinade}\label{peri-peri-marinade}}

\begin{itemize}
\tightlist
\item
  1 cup olive oil
\item
  \(\frac{1}{2}\) cup white vinegar
\item
  8 bird-eye chillis
\item
  8 cloves garlic
\item
  2 medium onions
\item
  4 teaspoons dried oregano
\item
  4 teaspoons paprika
\item
  2 teaspoons sugar
\item
  4 teaspoons salt
\item
  2 teaspoons pepper
\item
  1 roasted pepper
\item
  1 lemon, juiced
\end{itemize}

\hypertarget{chicken-burgers}{%
\section{\texorpdfstring{\href{https://www.foodnetwork.com/recipes/rachael-ray/bbq-chicken-burgers-with-slaw-recipe-1917239}{Chicken Burgers}}{Chicken Burgers}}\label{chicken-burgers}}

\hypertarget{paella}{%
\section{\texorpdfstring{\href{https://www.seriouseats.com/recipes/2019/09/stovetop-paella-mixta-for-two-with-chicken-and-shrimp.html}{Paella}}{Paella}}\label{paella}}

\emph{For the Sofrito}:

\begin{itemize}
\tightlist
\item
  3 dried ñora peppers or 4 ancho chilies (1 1/2 ounces total; 50g), optional; see note (Spanish dried ñora peppers add an earthy note to the sofrito; ancho chilies are a close approximation, though they have more heat. You can also omit the peppers entirely.)
\item
  1/4 cup (60ml) extra-virgin olive oil
\item
  3 medium cloves garlic, minced
\item
  2 medium yellow onions (3/4 pound; 300g), finely diced
\item
  One large (8-ounce/225g) red pepper, stemmed, seeded, and finely diced
\item
  Kosher salt
\item
  1 tablespoon (15ml) tomato paste
\end{itemize}

\emph{For the Paella}:
- 1 tablespoon (15ml) extra-virgin olive oil
- Kosher salt
- 2 bone-in, skin-on chicken thighs (1 pound; 450g)
- 1/3 cup sofrito (3 ounces; 85g)
- 1/4 teaspoon sweet smoked Spanish paprika (pimentón dulce)
- Pinch saffron threads
- 2 1/2 cups (590ml) boiling hot white chicken stock or low-sodium broth, vegetable stock, or water, plus more as needed
- 3/4 cup (5.25 ounces; 150g) short-grain Spanish rice, such as Bomba and Calasparra
- 6 large shelled shrimp
- Lemon wedges, for serving

\hypertarget{khanpunggi}{%
\section{\texorpdfstring{\href{https://www.maangchi.com/recipe/kkanpunggi/comment-page-3}{Khanpunggi}}{Khanpunggi}}\label{khanpunggi}}

\hypertarget{ingredients}{%
\subsection{Ingredients}\label{ingredients}}

\hypertarget{chicken}{%
\subsubsection{Chicken:}\label{chicken}}

\begin{itemize}
\item
  ½ pound chicken breast (about 230 grams), cut into bite size small pieces
\item
  ½ teaspoon minced ginger
\item
  1 teaspoon soy sauce
\item
  ¼ teaspoon ground black pepper
\item
  ½ cup potato starch
\item
  1 egg white (If you use 1 pound chicken, use 1 whole egg)
\item
  Spicy garlic- and leek-infused oil:
\item
  ¼ cup vegetable or corn oil
\item
  ½ cup thinly shredded leek
\item
  4 garlic cloves, cut into halves
\item
  1 tablespoon coarse red chili flakes
\end{itemize}

\hypertarget{vegetables-and-seasonings}{%
\subsubsection{Vegetables and seasonings:}\label{vegetables-and-seasonings}}

1 green chili pepper, deseeded, sliced thinly
1 fresh red chili pepper, deseeded, sliced thinly
1 green onion, chopped
½ medium sized onion, chopped
3-4 small dried red chili peppers
Sweet and sour sauce (all mixed together in a small bowl):

1 tablespoon soy sauce
2 tablespoons water
2 tablespoons rice syrup (or sugar)
1 tablespoon white vinegar
1 teaspoon potato starch
Oil:

1 cup cooking oil (grapeseed oil, vegetable oil, or corn oil)
1 teaspoon toasted sesame oil

\hypertarget{jerk-chicken}{%
\section{\texorpdfstring{\href{https://www.seriouseats.com/recipes/2013/08/jerk-chicken.html}{Jerk Chicken}}{Jerk Chicken}}\label{jerk-chicken}}

\hypertarget{overview}{%
\subsection{Overview}\label{overview}}

-\href{https://www.seriouseats.com/2013/08/the-food-lab-how-to-make-jerk-chicken-at-home.html}{\textbf{kenji's writedown}}
-\href{https://www.youtube.com/watch?v=Rt460jKi4Bk}{Another} on by Kwame Onwuachi, recipe \href{https://www.vice.com/en_us/article/akw3bb/the-best-jerk-chicken-recipe}{here}

\hypertarget{ingredients-1}{%
\subsection{ingredients}\label{ingredients-1}}

\begin{itemize}
\tightlist
\item
  6 whole Scotch bonnet peppers (see note)
\item
  6 scallions, roughly chopped
\item
  1 (2-inch) knob fresh ginger, roughly chopped
\item
  6 garlic cloves
\item
  2 tablespoons freshly picked thyme leaves
\item
  1 tablespoon ground allspice
\item
  1 teaspoon freshly grated nutmeg
\item
  2 tablespoons dark brown sugar
\item
  1/2 cup soy sauce
\item
  2 tablespoons zest and 1/4 cup juice from about 4 limes
\item
  1/4 cup olive oil
\item
  Kosher salt and freshly ground black pepper
\item
  1 large whole chicken, back removed, split in half along breastbone (4 to 4 1/2 pounds, see note)
\item
  1/4 cup whole allspice berries
\item
  3 dozen dried bay leaves (about 2 loosely packed cups)
\end{itemize}

\hypertarget{directions}{%
\subsection{Directions}\label{directions}}

\begin{enumerate}
\def\labelenumi{\arabic{enumi}.}
\item
  Combine peppers, scallions, ginger, garlic, thyme, allspice, nutmeg, brown sugar, soy sauce, lime zest and juice, olive oil, 2 teaspoons black pepper, and 1 tablespoon kosher salt in the work bowl of a food processor or the jar of a blender. Blend until a rough purée is formed, about 1 minute.
\item
  Place chickens in a large bowl or baking dish. Pour marinade over chickens and turn until thoroughly coated. Divide chicken and marinade between two gallon-sized zipper-lock bags, or place in a large baking dish and cover tightly with plastic wrap. Place whole allspice berries and bay leaves in a gallon-sized zipper-lock bag and fill with water. Refrigerate chicken and bay leaves at least 10 hours and up to 1 day.
\item
  When ready to cook, remove chicken from bags, allow excess marinade to drip off, and transfer to a large plate. Light one half chimney full of charcoal. When all the charcoal is lit and covered with gray ash, pour out and pile the coals against one wall of a kettle grill. Alternatively, set the leftmost burners of a gas grill to medium-high heat. Set cooking grate in place, cover grill and allow to preheat for 5 minutes. Clean and oil the grilling grate. Set bottom and lid vents to half open.
\item
  Drain bay leaves and allspice berries in a fine mesh strainer. Spread 2/3rds of bay leaves evenly over the cooler side of the grill (it's ok if some allspice berries fall through) in a pattern just large enough to fit the chickens. Lay the chickens over the bay leaves skin side up with the legs pointed towards the hotter side of the grill. Place 1/3 of remaining bay leaves over hot side of grill and immediately cover, with the vent above the chicken. Cook for 15 minutes.
\item
  Open lid and place half of remaining bay leaves and allspice berries on hot side of grill directly above the coals. Immediately cover and cook for another 15 minutes. Open lid, add 15 new coals to the pile of hot coals, then place remaining bay leaves and allspice berries on hot side of grill directly above the coals. Cover and continue to cook until the coolest part of the chicken breast registers 145°F on an instant read thermometer, about 20 minutes longer.
\item
  Uncover grill and wait five minutes until coals are hot again (if using gas grill, increase heat to high). Carefully lift the chicken off the bay leaves and transfer it to the hot side of the grill skin side up. Using tongs, drop the bay leaves into the grill directly onto the coals or burners so that they smoke. Cook the chicken until lightly charred, about 3 minutes. Flip chicken and continue to cook until skin is crisp and charred and coolest part of breast registers 150 to 155°F on an instant read thermometer, 4 to 6 minutes longer. Transfer to a large platter, allow to rest 5 minutes, and serve.
\end{enumerate}

\hypertarget{panang-curry}{%
\section{Panang curry}\label{panang-curry}}

1 1/2 lbs chicken thigh boneless, cut into bite size pieces
1 white onion finely chopped
5 cloves garlic minced
1 green bell peppers chopped
1 red bell peppers chopped
1/2 inch galangal roughly sliced
1 inch lemongrass roughly sliced
1 tablespoon coconut oil small jar Thai Panang curry paste measured to about 4 ½ tablespoons
1 tablespoon unsweetened peanut butter
2 teaspoon fish sauce
1 pinch nutmeg powder optional
Salt to taste if required
2 teaspoon palm sugar or brown sugar
6-8 kaffir lime leave crushed
2 cups coconut milk thick
1/4 cup Thai basil leaves

\hypertarget{for-the-paste}{%
\subsection{for the paste}\label{for-the-paste}}

INGREDIENTS
17 to 20 (2- to 3-inch-long) prik haeng (dried hot red chiles), halved and seeds discarded
4 teaspoons coriander seeds
2 fresh lemongrass stalks, 1 or 2 outer leaves discarded (or use reserved bottoms from iced lemongrass tea, page 160)
1 teaspoon whole black peppercorns
4 teaspoons finely chopped peeled fresh or thawed frozen greater galangal (sometimes called kha)
6 (4-inch-long) fresh or frozen Kaffir lime leaves (sometimes called bai makroot), finely chopped
2 tablespoons chopped fresh cilantro roots or stems
5 small shallots, chopped (6 tablespoons)
1/4 cup chopped garlic
15 to 20 (1-inch-long) red prik kii noo (fresh bird's-eye chiles) or serrano chiles, finely chopped
2 teaspoons ga-pi (Thai shrimp paste)
1/2 teaspoon salt

\hypertarget{teriyaki-glazed-salmon-cucumber-avocado-rice-bowel}{%
\section{\texorpdfstring{\href{https://www.seriouseats.com/recipes/2016/07/easy-teriyaki-glazed-salmon-cucumber-avocado-rice-bowl-recipe.html}{Teriyaki Glazed Salmon Cucumber Avocado Rice Bowel}}{Teriyaki Glazed Salmon Cucumber Avocado Rice Bowel}}\label{teriyaki-glazed-salmon-cucumber-avocado-rice-bowel}}

\begin{itemize}
\tightlist
\item
  4 salmon fillets, about 5 ounces (140g) each
\item
  Kosher salt and freshly ground black pepper
\item
  1 tablespoon (15ml) vegetable or canola oil
\item
  4 cups cooked white or brown rice (about 680g cooked rice)
\item
  1 avocado, diced
\item
  1 Persian or Japanese cucumber, diced
\item
  6 to 8 scallions, thinly sliced
\item
  1/2 cup (120ml) homemade or store-bought teriyaki sauce
\item
  Furikake and/or toasted sesame seeds, for serving (see note)
\end{itemize}

\hypertarget{easy-chicken-avocado-and-fried-egg-sandwich-recipe}{%
\section{\texorpdfstring{\href{https://www.vice.com/en_us/article/gyam9j/easy-chicken-avocado-and-fried-egg-sandwich-recipe}{Easy Chicken, Avocado, and Fried Egg Sandwich Recipe}}{Easy Chicken, Avocado, and Fried Egg Sandwich Recipe}}\label{easy-chicken-avocado-and-fried-egg-sandwich-recipe}}

\begin{itemize}
\tightlist
\item
  8 boneless and skinless chicken thighs
\item
  3 lemons
\item
  3/4 cup plus 2 tablespoons grapeseed oil
\item
  1 tablespoon poultry seasoning
\item
  2 teaspoons paprika
\item
  1 1/2 teaspoons cayenne
\item
  5 sprigs thyme, leaves picked and chopped, plus 2 whole sprigs
\item
  3 sprigs rosemary, leaves picked and finely chopped, plus 1 whole sprig
\item
  kosher salt and freshly ground black pepper, to taste
\item
  4 garlic cloves, peeled and smashed
\item
  1 jalapeño, stemmed and thinly sliced
\item
  1 medium red onion, thinly sliced
\item
  2 avocados, halved, pitted, and scooped
\item
  4 large eggs
\item
  1 pound manchego, grated
\item
  1 loaf 7-grain bread, sliced into 8 (½-inch) thick slices
\end{itemize}

\hypertarget{shakshuka}{%
\section{Shakshuka}\label{shakshuka}}

\begin{itemize}
\tightlist
\item
  3 tablespoons extra-virgin olive oil
\item
  1 large onion, halved and thinly sliced
\item
  1 large red bell pepper, seeded and thinly sliced
\item
  3 garlic cloves, thinly sliced
\item
  1 teaspoon ground cumin
\item
  1 teaspoon sweet paprika
\item
  ⅛ teaspoon cayenne, or to taste
\item
  1 (28-ounce) can whole plum tomatoes with juices, coarsely chopped
\item
  ¾ teaspoon salt, more as needed
\item
  ¼ teaspoon black pepper, more as needed
\item
  5 ounces feta cheese, crumbled (about 1 1/4 cups)
\item
  6 large eggs
\item
  Chopped cilantro, for serving
\item
  Hot sauce, for serving
\end{itemize}

\hypertarget{bbq-chicken-wings}{%
\section{BBQ Chicken Wings}\label{bbq-chicken-wings}}

\hypertarget{paste}{%
\subsection{PASTE}\label{paste}}

\begin{itemize}
\tightlist
\item
  1 medium onion , peeled and quartered
\item
  10 cloves garlic , peeled
\item
  2 fresh red chillies , stalks removed
\item
  olive oil
\end{itemize}

\hypertarget{herbs-and-spices}{%
\subsection{HERBS AND SPICES}\label{herbs-and-spices}}

\begin{itemize}
\tightlist
\item
  10 sprigs fresh thyme or lemon thyme , leaves picked
\item
  10 sprigs fresh rosemary , leaves picked
\item
  1 small bunch fresh coriander
\item
  10 bay leaves
\item
  1 teaspoon cumin seeds
\item
  2 tablespoons fennel seeds
\item
  2 tablespoons smoked paprika
\item
  6 cloves
\end{itemize}

\hypertarget{to-finish}{%
\subsection{TO FINISH}\label{to-finish}}

\begin{itemize}
\tightlist
\item
  2 oranges , zest and juice of
\item
  200 g soft brown sugar
\item
  6 tablespoons balsamic vinegar
\item
  200 ml tomato ketchup
\item
  2 tablespoons Worcestershire sauce
\item
  2 teaspoons English mustard
\item
  200 ml apple juice
\end{itemize}

\hypertarget{tandoor-chicken}{%
\section{\texorpdfstring{\href{https://www.vahrehvah.com/tandoori-chicken}{Tandoor Chicken}}{Tandoor Chicken}}\label{tandoor-chicken}}

\hypertarget{ingredients-2}{%
\subsection{Ingredients}\label{ingredients-2}}

\begin{itemize}
\tightlist
\item
  Ginger garlic paste - 1 tablespoon.
\item
  Garam masala powder - 1/2 tea spoon.
\item
  Cumin powder - 1 tablespoon.
\item
  Red chili powder - 2 tablespoons.
\item
  Red color water - 1 tea spoon.
\item
  Salt - to taste.
\item
  Turmeric powder - 1/4 tea spoon.
\item
  Yogurt - 6 tablespoons.
\item
  Chat masala - 1 tea spoon.
\item
  Coriander powder - 1 tea spoon.
\item
  Chicken legs - 6 numbers.
\item
  Kasuri methi powder (dry fenu greek leaves powder) - 1/2 tea spoon.
\item
  Lemon juice - 1 number.
\item
  Oil - 2 tablespoons.
\item
  Pepper powder - 1/2 tea spoon.
\item
  Mixed vegetable - 150 grams.
\end{itemize}

\hypertarget{method}{%
\subsection{Method}\label{method}}

\begin{itemize}
\tightlist
\item
  Clean and cut 2 or 3 long slits on each piece.
\item
  Apply salt, chili powder and 1/2 lime juice all over the chicken and keep aside for 15 minutes.
\item
  Make marination with coriander powder, cumin powder, red chilies, kasuri methi, turmeric powder, garam masala powder, red color, salt and mix well with yogurt.
\item
  Apply it all over the chicken making sure to apply well between all the slits and inside.
\item
  Preheat your oven to 425-degreesand cook for 25 to 35 minutes till the chicken is tender.
\item
  Remove from oven and serve hot.
\item
  garnished with sliced onions and lime wedges and also can be heated on a griddle if serving later.
\end{itemize}

\hypertarget{smacked-cucumber}{%
\section{Smacked Cucumber}\label{smacked-cucumber}}

\begin{itemize}
\tightlist
\item
  ginger
\item
  garlic
\item
  Sesame seed
\item
  soy
\item
  chili oil
\end{itemize}

\hypertarget{poblano-mole}{%
\section{\texorpdfstring{\href{https://www.seriouseats.com/recipes/2012/10/mole-poblano-recipe-how-to-make-mole.html}{Poblano Mole}}{Poblano Mole}}\label{poblano-mole}}

\hypertarget{roasted-doner-kebab}{%
\section{\texorpdfstring{\href{https://www.youtube.com/watch?v=TNChsYNpV0U}{Roasted ``Doner'' Kebab}}{Roasted ``Doner'' Kebab}}\label{roasted-doner-kebab}}

\hypertarget{fried-chicken}{%
\section{\texorpdfstring{\href{https://cravingsbychrissyteigen.com/watch/johns-legendary-fried-chicken-with-spicy-honey-butter/}{Fried Chicken}}{Fried Chicken}}\label{fried-chicken}}

\hypertarget{chicken-pho-pho-ga-in-instapot}{%
\section{Chicken Pho (Pho Ga in InstaPot)}\label{chicken-pho-pho-ga-in-instapot}}

\hypertarget{ingredients}{%
\subsection{Ingredients:}\label{ingredients}}

\begin{itemize}
\tightlist
\item
  2 tablespoons canola or vegetable oil
\item
  2 medium yellow onions, split in half
\item
  1 small hand of ginger, split in half
\item
  1 small bunch cilantro
\item
  3 star anise pods
\item
  1 cinnamon stick
\item
  4 cloves
\item
  1 teaspoon fennel seeds
\item
  1 teaspoon coriander seeds
\item
  6 to 8 chicken drumsticks
\item
  1/4 cup fish sauce, plus more to taste
\item
  2 tablespoons rock sugar or raw sugar, plus more to taste
\end{itemize}

\hypertarget{to-serve}{%
\subsection{To Serve:}\label{to-serve}}

\begin{itemize}
\tightlist
\item
  4 servings pho noodles, prepared according to package directions
\item
  1 small white or yellow onion, thinly sliced
\item
  1/2 cup thinly sliced scallions
\item
  2 cups mixed herbs (cilantro, basil, and mint)
\item
  2 cups trimmed bean sprouts
\item
  Thinly sliced Thai chilis
\item
  2 limes, each cut into 4 wedges
\item
  Hoisin sauce and Sriracha
\end{itemize}

\hypertarget{equipment}{%
\subsection{Equipment:}\label{equipment}}

\begin{itemize}
\tightlist
\item
  Instant Pot
\end{itemize}

\hypertarget{recipe}{%
\subsection{Recipe:}\label{recipe}}

\begin{enumerate}
\def\labelenumi{\arabic{enumi}.}
\tightlist
\item
  Char halved onions and gingers in broiler
\item
  Heat oil in the pot, then cilantro, star anise, cinnamon, cloves, fennel seed, coriander, and chicken to the pot. Brown the chicken in the pot. Add 2 quarts of water, the fish sauce, and the sugar to the pot. Seal the pressure cooker and bring it to high pressure over high heat. Cook on high pressure for 20 minutes.
\item
  Open pressure cooker. Transfer chicken legs to a plate. Pour broth through a fine mesh strainer into a clean pot and discard solids. Skim any scum off the surface of the broth using a ladle, but leave the small bubbles of fat intact. Season broth to taste with more fish sauce and sugar if desired.
\item
  To serve, place re-hydrated pho noodles in individual noodle bowls. Top with chicken legs, sliced onions, and scallions. Pour hot broth over chicken and noodles. Serve immediately, allowing guests to add herbs, bean sprouts, chilis, lime, and sauces as they wish.
\end{enumerate}

\hypertarget{thai-basil-chicken}{%
\section{Thai Basil Chicken}\label{thai-basil-chicken}}

\hypertarget{ingredients}{%
\subsection{Ingredients:}\label{ingredients}}

\begin{itemize}
\tightlist
\item
  4 minced chicken thighs
\item
  half an onion or shallot thinly sliced
\item
  2 red thai chilis sliced
\item
  1 cup of green beans cut into little cubes
\item
  3 cloves garlic minced
\item
  1 cup basil thinly sliced
\end{itemize}

\hypertarget{for-the-sauce}{%
\subsubsection{For the sauce:}\label{for-the-sauce}}

\begin{itemize}
\tightlist
\item
  1 cup chicken broth
\item
  2 tablespoons hoisin sauce
\item
  2 tablespoons soy sauce
\item
  1/8 teaspoon red chilli powder
\item
  2 - 3 teaspoons sugar
\end{itemize}

\hypertarget{directions}{%
\subsection{Directions:}\label{directions}}

\begin{enumerate}
\def\labelenumi{\arabic{enumi}.}
\tightlist
\item
  Combine ingredients for the sauce into a bowl, then set aside
\item
  Into a large skillet add 3 tablespoons olive oil, add chicken when skillet is hot and cook thouroughly
\item
  When chicken is cooked through, add onion, red thai chilis, green beans, and garlic and continue cooking until the water in skillet dries
\item
  Add half of the sauce and continue cooking until the sauce dries in skillet
\item
  Add the rest of the sauce and continue tossing the chicken
\item
  Once all the sauce is evaporated, add basil and stir
\end{enumerate}

\hypertarget{part-sous-video}{%
\part{Sous Video}\label{part-sous-video}}

\begin{itemize}
\tightlist
\item
  \href{https://www.chefsteps.com/activities/you-can-cook-frozen-food-sous-vide-without-defrosting-here-s-how}{You Can Cook Frozen Food Sous Vide Without Defrosting! Here's How}
\item
  \href{https://www.chefsteps.com/activities/sous-vide-time-and-temperature-guide}{Sous Vide Cooking Times}
\end{itemize}

\hypertarget{part-lamb-recipes}{%
\part{Lamb Recipes}\label{part-lamb-recipes}}

\hypertarget{spicy-cumin-lamb-noodles}{%
\section{Spicy Cumin Lamb Noodles}\label{spicy-cumin-lamb-noodles}}

Source: \url{https://ladyandpups.com/2015/03/04/my-xian-famous-spicy-cumin-lamb-hand-smashed-noodles/}.

\hypertarget{ingredients}{%
\subsection{Ingredients}\label{ingredients}}

\hypertarget{lamb-and-seasonings}{%
\subsubsection{Lamb And Seasonings:}\label{lamb-and-seasonings}}

\begin{enumerate}
\def\labelenumi{\arabic{enumi}.}
\tightlist
\item
  2 tbsp cumin seeds, toasted and coarsely ground
\item
  8.1 oz (230 grams) lamb, sliced
\end{enumerate}

\hypertarget{marinates}{%
\paragraph{Marinates:}\label{marinates}}

\begin{enumerate}
\def\labelenumi{\arabic{enumi}.}
\tightlist
\item
  1/2 tbsp soy sauce
\item
  1 tsp coarsely ground cumin (from above)
\item
  1 tsp corn starch
\item
  1/2 tsp extra dark soy sauce (for color)
\item
  1/2 tsp ground coriander
\item
  1/2 tsp chili flakes
\item
  1/2 tsp toasted sesame oil
\item
  1/4 tsp garlic powder
\item
  1/4 tsp ground white pepper
\item
  1/8 tsp ground sichuan peppercorn
\end{enumerate}

\hypertarget{seasoning-a}{%
\paragraph{Seasoning A:}\label{seasoning-a}}

\begin{itemize}
\tightlist
\item
  1/2 medium red onion, sliced
\item
  1 cup bean sprouts
\end{itemize}

\hypertarget{seasoning-b}{%
\paragraph{Seasoning B:}\label{seasoning-b}}

\begin{itemize}
\tightlist
\item
  4 tbsp canola oil
\item
  4 cloves garlic, finely minced
\item
  2 tsp grated ginger
\item
  1 large Asian red chili, diced (not spicy)
\item
  1 1/2 tbsp coarsely ground cumin (from above)
\item
  1 tsp ground coriander
\item
  1 tsp ground cayenne
\item
  1/2 tsp ground white pepper
\item
  1/4 tsp ground black pepper
\item
  1/8 tsp ground sichuan peppercorn
\end{itemize}

\hypertarget{seasoning-c}{%
\paragraph{Seasoning C:}\label{seasoning-c}}

\begin{itemize}
\tightlist
\item
  2 tbsp soy sauce
\item
  1 tbsp rice wine, or sake
\item
  1/2 tsp rice wine vinegar
\item
  1/4 tsp light brown sugar
\item
  1/4 tsp MSG
\item
  1/8 tsp salt
\end{itemize}

\hypertarget{seasoning-d}{%
\paragraph{Seasoning D:}\label{seasoning-d}}

\begin{itemize}
\tightlist
\item
  1/4 cup chopped fresh cilantro
\item
  1 tbsp chopped fresh mint
\item
  Serve with the best chili oil ever
\end{itemize}

\hypertarget{hand-smashed-noodle-strongly-recommend-measuring-by-weight}{%
\subsubsection{Hand-Smashed Noodle: (strongly recommend measuring by weight)}\label{hand-smashed-noodle-strongly-recommend-measuring-by-weight}}

\begin{itemize}
\tightlist
\item
  218 grams (1 1/2 cup) Chinese dumpling flour, or bread flour
\item
  2 grams (1/4 tsp) salt
\item
  126 grams (1/2 cup) water + 15 grams (1 tbsp) for adjustment
\end{itemize}

\hypertarget{recipe}{%
\subsection{Recipe}\label{recipe}}

Toast the cumin seeds on a skillet over medium heat, stirring constantly, until they start to pop and smell fragrant. Immediately transfer to a stone-mortar or spice-grinder before they burn. Grind them into a consistency that resembles coarsely ground black pepper, then set aside.

TO PREPARE THE LAMB AND SEASONINGS: Scatter lamb-slices flat on a chopping board in 1 single layer, then ``tap'' them all over with a sharp knife, aiming at scoring/tenderizing the meat without cutting through. Do this thoroughly. It allows the marinate to penetrate, and gives the lamb a more interesting texture. Then mix the lamb with the ``marinates'', using your hands to really distribute the seasonings evenly. Let marinate for at least 2 hours.

In 4 separate bowls, combine all the ingredients in each individual ``Seasoning A'', ``Seasoning B'', ``Seasoning C'' and ``Seasoning D''. Set aside.

TO MAKE THE HAND-SMASHED NOODLE: In a stand-mixer bowl (or with a hand-held mixer if it comes with dough-hooks)(hand-kneading not recommended), add Chinese dumpling flour, salt and water. Start mixing on low then gradually increase the speed to high, and knead for 5 \textasciitilde{} 6 min. The dough will feel shaggy and a bit dry in the beginning, but as the flour absorbs water and glutens start to form, it will become extremely smooth and elastic at the end. It will be sticky but pulls away cleanly from the bowl during mixing. You should be able to ``tap'' the dough quickly with your finger without it sticking, and pull it slowly upward into 12″ (30 cm) long without breaking. If the dough breaks, either it's not kneaded sufficiently or it's too dry. You'll have to try both ways (try kneading it for another 3 min first, before adding more water) to get it to the correct consistency. Cover the bowl with plastic wrap, and let rest for at least 1 hour.

Line a large baking sheet with parchment, set aside. Prepare a small cup of canola oil within your reach. Oil your hands well, then transfer the dough onto an oiled surface. Roll out into an approximately 1/2″ (1 cm) thick, rectangular shape, then cut into 10 long strips. Separate and lightly oil each strips so they don't stick back to each other. Take 1 strip and lay flat on the counter, then with oiled palm, start smashing/pounding the strip outward into a long, wide and flat noodle. Don't worry about evenness or straight edges because it doesn't matter. Now pick up the noodle on both ends, lift it and gently tap it on the counter while stretching it out slightly. You don't have to try hard. The noodle WANTS to stretch out and gravity will pretty much do the job for you! Lay the noodle flat, without any foldings, on the parchment-lined baking-sheet. Repeat with the rest (lay a new parchment over the top once you run out of space).

Carefully not to make the noodles too thin or they will lose their desired texture (you shouldn't be able to see through it).

TO COOK: Bring a large pot of water to boil. Meanwhile, heat another large deep skillet/wok over high heat. Mix 2 tbsp of canola oil (not in the ingredient list) into the marinated lamb to lubricate/separate them, then add to the hot skillet as spread out as possible. Let caramelize for 30 sec without moving, then start sautéing just until they are no longer pink. Add ``Seasoning A'' and cook just until it starts to soften, then transfer to a bowl.

Add 4 tbsp of canola oil from ``Seasoning B'' to the same skillet until hot, then add the rest of ``Seasoning B''. Cook until fragrant without burning the garlic, then add ``Seasoning C''. Turn off the heat while you cook the noodle. Add the noodles, one by one, into the boiling water. Cook just until they float to the surface (it will take less than a min), then drain/transfer to the skillet. Turn the skillet heat back on high, then add the lambs/onion and gently toss everything together. Finish with ``Seasoning D''.

Serve immediately with chili oil, and I'd like to sprinkle a bit more ground cumin on top.

\bibliography{book.bib}

\printindex

\end{document}
